\documentclass[letterpaper,12pt]{article}
\usepackage[utf8]{inputenc}
\begin{document}

\title{Redes temáticas en investigación genómica. Desarrollo de herrmientas de análisis estratégico}
\maketitle

\section{Tema central}

El tema central del proyecto son las relaciones temáticas vinculados con la genómica. Las relaciones entre los temas serán modeladas y analizadas, así como se desarrollarán herramientas para su visualización permitiendo de este modo una aproximación más intuitiva a la estructura histórica y actual del conocimiento genómico en contexto. 

Este proyecto es relevante porque permitirá producir herramientas que generen marcos de referencia temáticos, basados en las publicaciones relativas a la genómica y temas vinculados, en los cuales se podrá ubicar el trabajo que desarrollan los investigadores del INMEGEN. Asimismo permitirá identificar áreas de oportunidad en investigación.

\section{Objetivos}

\subsection{Objetivo general}
Los objetivos generales del proyecto son dos:

\begin{itemize}

\item {Desarrollar software como una herramienta de análisis estratégico a través de la modelización de las relaciones entre los temas de investigación vinculados con la genómica.}

\item {Entender la estructura y dinámica generadas por las relaciones entre los temas de investigación vinculados con la genómica.}

\end{itemize}

\subsection{Objetivos particulares}

\begin{itemize}

\item{Modelar y analizar las redes temáticas}

\item{Desarrollar herramientas de visualización de las redes}

\item{Analizar la topología de las redes}

\item{Identificar \emph{Trending topics}}

\item{Desarrollar herramientas para predecir \emph{Trending topics}}

\item{Identificar los vínculos entre \emph{Trending topics} y la emergencia de disciplinas en genómica}

\end{itemize}

\section{Metodología}

La metodología consistirá en:

\begin{itemize}

\item{Minería de datos}

\item{Minería de textos}

\item{Modelado y análisis de redes basados en Cytoscape y NetworkX}

\item{Visualización de redes con OpenGL}

\end{itemize}

Todo el desarrollo de software estará versionado y con control de cambios.

\end{document}

